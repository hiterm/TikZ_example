% for pdflatex
% data.py を実行してからこちらをタイプセット
\documentclass[border=1pt]{standalone}

\usepackage[T1]{fontenc}
\usepackage{lmodern}

\usepackage{tikz}
\usetikzlibrary{calc}
\usetikzlibrary{plotmarks}

\begin{document}

\begin{tikzpicture}
  \newcommand{\p}{11}
  \newcommand{\q}{7}
  \everymath{\displaystyle}
  \coordinate (A) at (0,0);
  \coordinate (B) at (\p,0);
  \coordinate (D) at (0,\q);
  \coordinate (C) at ($(B) + (D)$); %足し算
  \coordinate (X) at ($(A)!0.5!(B)$); %中点
  \coordinate (Z) at ($(C)!0.5!(D)$);
  \coordinate (Y) at ($(X)!0.5!(Z)$);
  \coordinate (W) at ($(A)!0.5!(D)$);
  \draw [->] (A) -- ($(A)!1.1!(B)$);
  \draw [->] (A) -- ($(A)!1.1!(D)$);
  \draw (A) node[below left] {A} -- (B) node[above right] {B} % node[below] {$p$}
    -- (C) node[above] {C} -- (D) node[above right] {D} % node[left] {$q$}
    -- cycle;
  \draw (A) -- (C);
  \draw [dashed] (X) node[above right] {X} % node[below] {$\frac{p}{2}$}
    -- (Z) node[above] {Z};
  \draw [dashed] (W) node[above right] {W} % node[left] {$\frac{q}{2}$}
    -- (Y) node[above left] {Y};

  \foreach \i in {1,...,\p}{
    \draw (\i, 0) node[below] {$\i$} -- (\i, 0.2);
  }
  \foreach \j in {1,...,\q}{
    \draw (0, \j) node[left] {$\j$} -- (0.2, \j);
  }

  \draw plot [mark=*, only marks] file {eisenstein-1.table};
  \draw plot [mark=x, only marks, mark size=3pt] file {eisenstein-2.table};
  \draw plot [mark=triangle, only marks, mark size=3pt] file {eisenstein-3.table};
\end{tikzpicture}

\end{document}
